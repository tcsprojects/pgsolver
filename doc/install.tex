\section{Installation Guide}

\subsection{Obtaining the Relevant Parts}

You can obtain the source code of the \pgsolver library from
\begin{verbatim}
    http://www.tcs.ifi.lmu.de/pgsolver
\end{verbatim}
Download and unpack the archive.
\begin{verbatim}
    ~> tar xzvf pgsolver.tgz
\end{verbatim}
% name of the actual archive you have obtained may vary in the version number.
Unpacking will create a directory \texttt{pgsolver} and various
subdirectories in it.

In order to compile \pgsolver from source code you will need
\begin{itemize}
\item the OCaml compiler; we recommend version 3.09.2 or higher, but earlier versions
      may just as well be fine. You can get it from
      \begin{verbatim}
    http://caml.inria.fr\end{verbatim}
\item GNU make; we recommend version 3.81 or higher but earlier ones may suffice as well.
      You can get it from
      \begin{verbatim}
    http://www.gnu.org/software/make\end{verbatim}
\item (optionally) one ore more SAT solvers. Note that this is only necessary if you want
      \pgsolver to also contain the algorithms relying on SAT solving backends. You can
      compile it without. Although it should be possible to integrate almost every linkable
      SAT solver, there are three SAT solver backends that are already supported by
      \pgsolver:

	\begin{itemize}
		\item the SAT solver \texttt{zChaff} developed in Princeton. It is available from
        \begin{verbatim}
      http://www.princeton.edu/~chaff/zchaff.html\end{verbatim}
        We recommend version 2007.3.12. Earlier version will probably not suffice.

        \item the SAT solver \texttt{PicoSAT} developed in Linz. It is available from
        \begin{verbatim}
      http://fmv.jku.at/picosat\end{verbatim}
        We recommend version 632. Earlier version will probably not suffice.

        \item the SAT solver \texttt{MiniSat} developed in G\"oteborg. It is available from
        \begin{verbatim}
      http://minisat.se\end{verbatim}
        We recommend the C-version of 1.14. Other version will probably not work with our interface.
    \end{itemize}

\item (optionally) a C++ compiler like \texttt{g++} for example. This is only necessary
      to compile \texttt{zChaff} and the parts of \pgsolver that link it. You can get it
      from
      \begin{verbatim}
    http://gcc.gnu.org\end{verbatim}
\end{itemize}
We assume that the OCaml compiler as well as GNU make are installed. A quick installation
guide for the mentioned SAT solvers is provided below. If you want \pgsolver to contain the
reductions to SAT you obviously need to install at least one SAT solver before compiling \pgsolver.

\subsection{Compiling \pgsolver}

Now change into the directory created by unpacking the \pgsolver source code.
\begin{verbatim}
    ~> cd pgsolver
\end{verbatim}
The first (and probably most important) step consists of adjusting the \texttt{Config} files.
There are two config files, \texttt{./Config.default} and \texttt{./satsolvers/Config.default},
that have to be edited. It is highly recommended to create a copy of both files with the name \texttt{Config}
in the respective directory that is to be edited instead of the original versions. The Makefile
checks whether a customized configuration file named \texttt{Config} exists and if so it is used
instead of the default versions.

Both configuration files start with declarations about where to find all the programs necessary to build
the executable \pgsolver.
\begin{verbatim}
    TAR=tar
    OCAMLOPT=ocamlopt
    OCAMLLEX=ocamllex
    OCAMLYACC=ocamlyacc
    CPP=g++
    OCAMLOPTCPP=g++
\end{verbatim}
Change these lines to point to the full path in which the OCaml compiler, lexer and parser
generator are installed unless they are in the current PATH. The lines pointing to the C++
compiler only need to be configured if there is at least one sat solver that is to be linked
with \pgsolver.

You need to give the full path of you OCaml installation directory.
\begin{verbatim}
    OCAML_DIR=/usr/lib/ocaml
\end{verbatim}


For each supported SAT solver, there are some lines that need to be configured in order
to use the include the respective SAT solver.
\begin{verbatim}
    WITH_RESPECTIVE_SAT_SOLVER=YES
    PATH_TO_OBJECT_FILE=...
\end{verbatim}
If you do want to have support for the reductions to SAT then make sure that at least one
SAT solver is properly configured and enabled.

Once you have adjusted both \texttt{Config} files accordingly you can now compile \pgsolver
by simply calling the make program.
\begin{verbatim}
    ~/pgsolver> make
\end{verbatim}
This is the same as
\begin{verbatim}
    ~/pgsolver> make pgsolver
\end{verbatim}
After successful compilation, the executable can be found as \texttt{pgsolver} in the
subdirectory \texttt{bin}.

You can delete all files that have been created during the compilation process by running
\begin{verbatim}
    ~/pgsolver> make clean
\end{verbatim}
If you ever encounter error message like
\begin{verbatim}
    The files obj/paritygame.cmi and obj/transformations.cmi
    make inconsistent assumptions over interface Paritygame
\end{verbatim}
during the compilation process, which should only occur when the source code has been changed, then
use this to get rid of them.

If you also want to have executable programs that create the benchmarks described in the next
chapter as well as some possibly useful tools that massage parity games then run
\begin{verbatim}
    ~/pgsolver> make generators
\end{verbatim}
and
\begin{verbatim}
    ~/pgsolver> make tools
\end{verbatim}
Finally,
\begin{verbatim}
    ~/pgsolver> make all
\end{verbatim}
is a synonym for \verb#make pgsolver; make generators; make tools#.


\subsection{Integrating SAT solvers}

\pgsolver currently supports three backend SAT solvers, namely \texttt{zChaff}, \texttt{PicoSAT} and
\texttt{MiniSat}. It should be also possible to integrate other SAT solvers; see the Developer's Guide for
more information on that subject. In order to integrate one of them into \pgsolver, you need to follow these
steps:
\begin{enumerate}
\item Download and compile the respective SAT solver.
\item Adjust the \texttt{satsolvers/Config} file s.t.\ the usage of the respective SAT solver is enabled and all required links point to the correct target.
\item Remove the current compilation of \pgsolver: \verb#make cleanall#.
\item Recompile \pgsolver: \verb#make all#.
\end{enumerate}


\subsubsection{Compiling \texttt{zChaff}}

Obtain the source code of \texttt{zChaff} (we recommend at least version 2007.3.12), unpack it and
consult the included \texttt{README} file for instructions on how to compile it. If you are lucky
then a simple
\begin{verbatim}
    ~/zchaff> make
\end{verbatim}
will do the job.

% \hide{
% Check in the included \texttt{Makefile} or via the output produced during compilation which C++ compiler
% is being used. For example, in \texttt{Makefile} there should be a line of the form
% \begin{verbatim}
%     CC = g++ -Wall
% \end{verbatim}
% which tells you that the compiler used to build \texttt{zChaff} is \texttt{g++}. You will have to make
% sure that this is also the value of the variable \texttt{CPP} in \pgsolver's \texttt{Makefile}. Otherwise
% \texttt{zChaff} cannot be linked with the modules of the \pgsolver library.
% }

The compilation of \texttt{zChaff} produces an executable \texttt{zchaff}. However, \pgsolver uses the
library version which can be linked into the program directly. This is usually called \texttt{libsat.a}
and is also produced by \texttt{zChaff}'s compilation process. Make sure that the variable
\verb#ZCHAFFLIB# in \pgsolver's \texttt{satsolvers/Config} file points to the directory in which \texttt{libsat.a}
can be found, for example
\begin{verbatim}
    #######
    # ZCHAFF
    #######
    ZCHAFFLIB=/usr/local/lib/zchaff-2007-3-12/libsat.a
    WITH_ZCHAFF=YES
\end{verbatim}


\subsubsection{Compiling \texttt{PicoSAT}}

Obtain the source code of \texttt{PicoSAT} (we recommend at least version 632), unpack it and
consult the included \texttt{README} file for instructions on how to compile it. Usually,
\begin{verbatim}
    ~/picosat> ./configure && make
\end{verbatim}
will do the job.

The compilation of \texttt{PicoSAT} produces an executable \texttt{picosat}. However, \pgsolver uses the
object files which can be linked into the program directly. Make sure that the variable
\verb#PICOSATDIR# in \pgsolver's \texttt{satsolvers/Config} file points to the directory in which the object files
can be found, for example
\begin{verbatim}
    #######
    # PICOSAT
    #######
    PICOSATDIR=/usr/local/lib/picosat-632
    WITH_PICOSAT=YES
\end{verbatim}


\subsubsection{Compiling \texttt{MiniSat}}

Obtain the source code of \texttt{MiniSat} (we recommend the C-version 1.14), unpack it and
consult the included \texttt{README} file for instructions on how to compile it. Usually,
\begin{verbatim}
    ~/minisat/core> make
\end{verbatim}
will do the job.

The compilation of \texttt{MiniSat} produces an executable \texttt{minisat}. However, \pgsolver uses the
object files which can be linked into the program directly. Make sure that both
\verb#MINISAT# variables in \pgsolver's \texttt{satsolvers/Config} file point to the respective directories, for example
\begin{verbatim}
    #######
    # MINISAT
    #######
    MINISATDIR=/usr/local/lib/minisat/core
    MINISATMTL=/usr/local/lib/minisat/mtl
    WITH_MINISAT=YES
\end{verbatim}



%%% Local Variables:
%%% mode: latex
%%% TeX-master: "main"
%%% End:
